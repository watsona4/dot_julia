% vim:ft=tex:
%
\documentclass[12pt]{article}

\usepackage{amsmath}

\newcommand{\syncpar}{\mathsf{F}}
\newcommand{\rpole}{\varrho_\text{pole}}

\title{
	method
}
\author{
	Mark Wells --- \texttt{mwellsa@gmail.com}
}

\begin{document}
\maketitle

Generalized potential with respect to star 1 is given by Equation 3.16 in PHOEBE Scientific Reference
\begin{equation}
    \Omega = \frac{1}{\varrho} + q \left( \frac{1}{\sqrt{ \delta^2
                                                        + \varrho^2
                                                        - 2\varrho\lambda\delta
                                                        }
                                                  }
                                        - \frac{\varrho\lambda}{\delta^2}
                                   \right)
                                   + \frac{1}{2}\syncpar^2(1+q)\varrho^2(1-\nu^2)
\end{equation}
To get star 2 (Equation~3.15 in P.S.R.)
\begin{equation}
    \Omega' = \frac{\Omega}{q} + \frac{1}{2}\frac{q-1}{q}
\end{equation}
where
\begin{itemize}
    \item $\varrho = r/a$ where $r$ is the radial distance originating from star 1 and $a$ is the semi-major axis
    \item $q$ is the mass ratio
    \item $\delta = D/a$ is the normalized instantaneous separation between the two stars
    \item $\syncpar$ is the synchronicity parameter (will be introduced when needed)
    \item $\lambda$ and $\nu$ are the $x$ and $z$ directional cosines, respectively (Eqn.~3.8~P.S.R.)
\end{itemize}

The potential at the pole reduces to Equation~3.20 (PHOEBE Sci. Ref.)
\begin{equation}
    \Omega = \frac{1}{\rpole} + q\left(\frac{1}{\sqrt{\delta^2 + \rpole^2}}\right)
\end{equation}


The position of $L_1$ is given by Equation~3.23 (PHOEBE Sci. Ref.)
\begin{equation}
    x_{L1} = z - \frac{1}{3}z^2 - \frac{1}{9}z^3 + \frac{58}{81}z^4
\end{equation}
where $z = (\mu/3)^{1/3}$ and $\mu = M_2/(M_1 + M_2)$

Before plugging back in to the equation of the general potential lets note the following:
\begin{itemize}
    \item along the $x$-axis so $\nu = 0$ and $\lambda = 1$
    \item $\varrho = x_{L1}$
    \item lets check at periastron so $\delta = (1-\varepsilon)$
\end{itemize}
\begin{equation}
    \Omega = \frac{1}{x_{L1}} + q \left( \frac{1}{\sqrt{ \delta^2
                                                        + x_{L1}^2
                                                        - 2x_{L1}\delta
                                                        }
                                                  }
                                        - \frac{x_{L1}}{\delta^2}
                                   \right)
                                   + \frac{1}{2}\syncpar^2(1+q)x_{L1}^2
\end{equation}


We adopt the condition that
\begin{equation}
\syncpar = 
    \begin{cases}
        1 & \text{if } \varepsilon < 0.05 \\
        \sqrt{\frac{1+\varepsilon}{(1-\varepsilon)^3}} & \text{if } \varepsilon \geq 0.05
   \end{cases}
\end{equation}


\end{document}
